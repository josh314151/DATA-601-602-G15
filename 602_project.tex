% Options for packages loaded elsewhere
\PassOptionsToPackage{unicode}{hyperref}
\PassOptionsToPackage{hyphens}{url}
%
\documentclass[
]{article}
\usepackage{amsmath,amssymb}
\usepackage{iftex}
\ifPDFTeX
  \usepackage[T1]{fontenc}
  \usepackage[utf8]{inputenc}
  \usepackage{textcomp} % provide euro and other symbols
\else % if luatex or xetex
  \usepackage{unicode-math} % this also loads fontspec
  \defaultfontfeatures{Scale=MatchLowercase}
  \defaultfontfeatures[\rmfamily]{Ligatures=TeX,Scale=1}
\fi
\usepackage{lmodern}
\ifPDFTeX\else
  % xetex/luatex font selection
\fi
% Use upquote if available, for straight quotes in verbatim environments
\IfFileExists{upquote.sty}{\usepackage{upquote}}{}
\IfFileExists{microtype.sty}{% use microtype if available
  \usepackage[]{microtype}
  \UseMicrotypeSet[protrusion]{basicmath} % disable protrusion for tt fonts
}{}
\makeatletter
\@ifundefined{KOMAClassName}{% if non-KOMA class
  \IfFileExists{parskip.sty}{%
    \usepackage{parskip}
  }{% else
    \setlength{\parindent}{0pt}
    \setlength{\parskip}{6pt plus 2pt minus 1pt}}
}{% if KOMA class
  \KOMAoptions{parskip=half}}
\makeatother
\usepackage{xcolor}
\usepackage[margin=1in]{geometry}
\usepackage{color}
\usepackage{fancyvrb}
\newcommand{\VerbBar}{|}
\newcommand{\VERB}{\Verb[commandchars=\\\{\}]}
\DefineVerbatimEnvironment{Highlighting}{Verbatim}{commandchars=\\\{\}}
% Add ',fontsize=\small' for more characters per line
\usepackage{framed}
\definecolor{shadecolor}{RGB}{248,248,248}
\newenvironment{Shaded}{\begin{snugshade}}{\end{snugshade}}
\newcommand{\AlertTok}[1]{\textcolor[rgb]{0.94,0.16,0.16}{#1}}
\newcommand{\AnnotationTok}[1]{\textcolor[rgb]{0.56,0.35,0.01}{\textbf{\textit{#1}}}}
\newcommand{\AttributeTok}[1]{\textcolor[rgb]{0.13,0.29,0.53}{#1}}
\newcommand{\BaseNTok}[1]{\textcolor[rgb]{0.00,0.00,0.81}{#1}}
\newcommand{\BuiltInTok}[1]{#1}
\newcommand{\CharTok}[1]{\textcolor[rgb]{0.31,0.60,0.02}{#1}}
\newcommand{\CommentTok}[1]{\textcolor[rgb]{0.56,0.35,0.01}{\textit{#1}}}
\newcommand{\CommentVarTok}[1]{\textcolor[rgb]{0.56,0.35,0.01}{\textbf{\textit{#1}}}}
\newcommand{\ConstantTok}[1]{\textcolor[rgb]{0.56,0.35,0.01}{#1}}
\newcommand{\ControlFlowTok}[1]{\textcolor[rgb]{0.13,0.29,0.53}{\textbf{#1}}}
\newcommand{\DataTypeTok}[1]{\textcolor[rgb]{0.13,0.29,0.53}{#1}}
\newcommand{\DecValTok}[1]{\textcolor[rgb]{0.00,0.00,0.81}{#1}}
\newcommand{\DocumentationTok}[1]{\textcolor[rgb]{0.56,0.35,0.01}{\textbf{\textit{#1}}}}
\newcommand{\ErrorTok}[1]{\textcolor[rgb]{0.64,0.00,0.00}{\textbf{#1}}}
\newcommand{\ExtensionTok}[1]{#1}
\newcommand{\FloatTok}[1]{\textcolor[rgb]{0.00,0.00,0.81}{#1}}
\newcommand{\FunctionTok}[1]{\textcolor[rgb]{0.13,0.29,0.53}{\textbf{#1}}}
\newcommand{\ImportTok}[1]{#1}
\newcommand{\InformationTok}[1]{\textcolor[rgb]{0.56,0.35,0.01}{\textbf{\textit{#1}}}}
\newcommand{\KeywordTok}[1]{\textcolor[rgb]{0.13,0.29,0.53}{\textbf{#1}}}
\newcommand{\NormalTok}[1]{#1}
\newcommand{\OperatorTok}[1]{\textcolor[rgb]{0.81,0.36,0.00}{\textbf{#1}}}
\newcommand{\OtherTok}[1]{\textcolor[rgb]{0.56,0.35,0.01}{#1}}
\newcommand{\PreprocessorTok}[1]{\textcolor[rgb]{0.56,0.35,0.01}{\textit{#1}}}
\newcommand{\RegionMarkerTok}[1]{#1}
\newcommand{\SpecialCharTok}[1]{\textcolor[rgb]{0.81,0.36,0.00}{\textbf{#1}}}
\newcommand{\SpecialStringTok}[1]{\textcolor[rgb]{0.31,0.60,0.02}{#1}}
\newcommand{\StringTok}[1]{\textcolor[rgb]{0.31,0.60,0.02}{#1}}
\newcommand{\VariableTok}[1]{\textcolor[rgb]{0.00,0.00,0.00}{#1}}
\newcommand{\VerbatimStringTok}[1]{\textcolor[rgb]{0.31,0.60,0.02}{#1}}
\newcommand{\WarningTok}[1]{\textcolor[rgb]{0.56,0.35,0.01}{\textbf{\textit{#1}}}}
\usepackage{graphicx}
\makeatletter
\def\maxwidth{\ifdim\Gin@nat@width>\linewidth\linewidth\else\Gin@nat@width\fi}
\def\maxheight{\ifdim\Gin@nat@height>\textheight\textheight\else\Gin@nat@height\fi}
\makeatother
% Scale images if necessary, so that they will not overflow the page
% margins by default, and it is still possible to overwrite the defaults
% using explicit options in \includegraphics[width, height, ...]{}
\setkeys{Gin}{width=\maxwidth,height=\maxheight,keepaspectratio}
% Set default figure placement to htbp
\makeatletter
\def\fps@figure{htbp}
\makeatother
\setlength{\emergencystretch}{3em} % prevent overfull lines
\providecommand{\tightlist}{%
  \setlength{\itemsep}{0pt}\setlength{\parskip}{0pt}}
\setcounter{secnumdepth}{-\maxdimen} % remove section numbering
\ifLuaTeX
  \usepackage{selnolig}  % disable illegal ligatures
\fi
\IfFileExists{bookmark.sty}{\usepackage{bookmark}}{\usepackage{hyperref}}
\IfFileExists{xurl.sty}{\usepackage{xurl}}{} % add URL line breaks if available
\urlstyle{same}
\hypersetup{
  pdftitle={Data 602 - Project},
  pdfauthor={Josh Brauner, Raahim Salman, Ze Yu},
  hidelinks,
  pdfcreator={LaTeX via pandoc}}

\title{Data 602 - Project}
\author{Josh Brauner, Raahim Salman, Ze Yu}
\date{Feburary 14th, 2024}

\begin{document}
\maketitle

\hypertarget{part-1}{%
\section{Part 1}\label{part-1}}

\hypertarget{data-wrangling}{%
\subsection{Data Wrangling}\label{data-wrangling}}

In the initial phase of our project, the primary goal is to prepare the
dataset for analysis. This preparation involves loading the data and
then removing data points that could potentially skew or bias our
results. Specifically, the dataset encompasses annual home run rates
(home runs per at bat) for Barry Bonds, covering several seasons. Per
the project's guidelines, it's imperative to exclude the 2001
season---the year Barry Bonds hit a record 73 home runs---from our
dataset before proceeding with any further analysis. The rationale
behind excluding this particular season is rooted in its outlier status
within the scope of our investigation, possibly due to factors such as
intentional walks, which are not reflective of the typical performance
trends we aim to analyze.

\begin{Shaded}
\begin{Highlighting}[]
\NormalTok{bonds\_data }\OtherTok{\textless{}{-}} \FunctionTok{read.csv}\NormalTok{(}\StringTok{"bondsdata.csv"}\NormalTok{)}
\NormalTok{bonds\_data\_filtered }\OtherTok{\textless{}{-}} \FunctionTok{subset}\NormalTok{(bonds\_data, season }\SpecialCharTok{!=} \DecValTok{2001}\NormalTok{)}
\FunctionTok{tail}\NormalTok{(bonds\_data\_filtered)}
\end{Highlighting}
\end{Shaded}

\begin{verbatim}
##    season     hrat
## 9    1995 0.065217
## 10   1996 0.081238
## 11   1997 0.075188
## 12   1998 0.067029
## 13   1999 0.095775
## 14   2000 0.102083
\end{verbatim}

The removal of the 2001 season data point is a critical step in ensuring
a more accurate and unbiased evaluation of Bonds' performance trends
over the years leading up to this extraordinary season. By focusing on
the years preceding 2001, we aim to construct a statistical model that
predicts Bonds' home run rates without the influence of this anomalous
year. This approach is intended to provide a clearer perspective on his
performance trajectory, allowing us to assess whether there is evidence
of an unusual improvement that could be attributed to external factors,
such as the speculated use of steroids.

\hypertarget{model-building}{%
\subsection{Model Building}\label{model-building}}

The statistical model we plan to build,
\(HRAT_{i} = A + B \times Year_{i} + e_{i}\), where \(HRAT_{i}\)
represents the home run rate in year \(i\), and \(Year_{i}\) denotes the
year of the season, is designed to elucidate the relationship between
time (years) and Bonds' home run rates. Through quantifying the trend
over the specified years, this linear regression model serves as a tool
for analyzing potential significant deviations in performance. Such
deviations, if present, could align with steroid use, under the premise
that such use would manifest as an atypical increase in home run rates.

\begin{Shaded}
\begin{Highlighting}[]
\NormalTok{model }\OtherTok{\textless{}{-}} \FunctionTok{lm}\NormalTok{(hrat }\SpecialCharTok{\textasciitilde{}}\NormalTok{ season, }\AttributeTok{data =}\NormalTok{ bonds\_data\_filtered)}
\NormalTok{model}\SpecialCharTok{$}\NormalTok{coef}
\end{Highlighting}
\end{Shaded}

\begin{verbatim}
##  (Intercept)       season 
## -7.992499290  0.004044169
\end{verbatim}

The linear regression equation representing the model is
\(\hat{HRAT_{i}} = -7.992 + 0.004 \times Year_{i}\).

\begin{Shaded}
\begin{Highlighting}[]
\FunctionTok{ggplot}\NormalTok{(bonds\_data\_filtered, }\FunctionTok{aes}\NormalTok{(}\AttributeTok{x =}\NormalTok{ season, }\AttributeTok{y =}\NormalTok{ hrat)) }\SpecialCharTok{+}
  \FunctionTok{geom\_point}\NormalTok{() }\SpecialCharTok{+}
  \FunctionTok{geom\_smooth}\NormalTok{(}\AttributeTok{method =} \StringTok{"lm"}\NormalTok{, }\AttributeTok{se =} \ConstantTok{FALSE}\NormalTok{, }\AttributeTok{color =} \StringTok{"\#00bbff"}\NormalTok{) }\SpecialCharTok{+}
  \FunctionTok{labs}\NormalTok{(}\AttributeTok{title =} \StringTok{"Barry Bonds HR/AB vs. Season"}\NormalTok{, }\AttributeTok{x =} \StringTok{"Season"}\NormalTok{, }\AttributeTok{y =} \StringTok{"HR/AB"}\NormalTok{) }\SpecialCharTok{+}
  \FunctionTok{theme\_minimal}\NormalTok{()}
\end{Highlighting}
\end{Shaded}

\includegraphics{602_project_files/figure-latex/unnamed-chunk-4-1.pdf}

Importantly, to ensure the integrity of the conclusions drawn from this
model, we will verify the assumptions underlying linear regression
analysis, including linearity, independence of errors, homoscedasticity
(constant variance of errors), and normality of error terms. These
checks are fundamental to confirming that our model is accurately
specified and that the inferences and predictions derived from it are
reliable.

\hypertarget{model-assumption-and-validation}{%
\subsection{Model Assumption and
Validation}\label{model-assumption-and-validation}}

\textbf{\emph{Correlation Coefficient Check}}:

\begin{Shaded}
\begin{Highlighting}[]
\FunctionTok{cor}\NormalTok{(bonds\_data\_filtered}\SpecialCharTok{$}\NormalTok{season, bonds\_data\_filtered}\SpecialCharTok{$}\NormalTok{hrat)}
\end{Highlighting}
\end{Shaded}

\begin{verbatim}
## [1] 0.7981544
\end{verbatim}

A correlation coefficient of 0.7981544 indicates a strong positive
linear relationship between the year and the home runs per at bat
(HR/AB) for Barry Bonds, excluding the 2001 season.

\textbf{\emph{Significance of Coefficient Estimates}}

In the context of analyzing the significance of the coefficient estimate
for the year in predicting home runs per at bat (HR/AB) for Barry Bonds
(excluding the 2001 season), we can formulate the null hypothesis
(\(H_0\)) and the alternative hypothesis (\(H_1\)) as follows, with an
alpha level (\(\alpha\)) of 0.05:

Null hypothesis: \(H_{0}: \beta_{1} = 0\), \(H_{0}: \beta_{0} = 0\)

Alternative hypothesis: \(H_{A}: \beta_{1} \neq 0\),
\(H_{A}: \beta_{0} \neq 0\)

\begin{Shaded}
\begin{Highlighting}[]
\FunctionTok{summary}\NormalTok{(model)}
\end{Highlighting}
\end{Shaded}

\begin{verbatim}
## 
## Call:
## lm(formula = hrat ~ season, data = bonds_data_filtered)
## 
## Residuals:
##       Min        1Q    Median        3Q       Max 
## -0.020722 -0.009931  0.001841  0.007701  0.023055 
## 
## Coefficients:
##               Estimate Std. Error t value Pr(>|t|)    
## (Intercept) -7.9924993  1.7566775  -4.550 0.000666 ***
## season       0.0040442  0.0008812   4.589 0.000622 ***
## ---
## Signif. codes:  0 '***' 0.001 '**' 0.01 '*' 0.05 '.' 0.1 ' ' 1
## 
## Residual standard error: 0.01329 on 12 degrees of freedom
## Multiple R-squared:  0.6371, Adjusted R-squared:  0.6068 
## F-statistic: 21.06 on 1 and 12 DF,  p-value: 0.0006222
\end{verbatim}

\begin{Shaded}
\begin{Highlighting}[]
\FunctionTok{coef}\NormalTok{(}\FunctionTok{summary}\NormalTok{(model))[, }\StringTok{"t value"}\NormalTok{]}
\end{Highlighting}
\end{Shaded}

\begin{verbatim}
## (Intercept)      season 
##   -4.549782    4.589384
\end{verbatim}

\begin{Shaded}
\begin{Highlighting}[]
\FunctionTok{coef}\NormalTok{(}\FunctionTok{summary}\NormalTok{(model))[, }\StringTok{"Pr(\textgreater{}|t|)"}\NormalTok{]}
\end{Highlighting}
\end{Shaded}

\begin{verbatim}
##  (Intercept)       season 
## 0.0006664296 0.0006222474
\end{verbatim}

\hypertarget{intercept}{%
\subsubsection{Intercept}\label{intercept}}

\begin{itemize}
\tightlist
\item
  \textbf{t-value for Intercept}: -4.549782
\item
  \textbf{p-value for Intercept}: 0.0006664296
\end{itemize}

The intercept's t-value is significantly negative, and the corresponding
p-value is much less than the alpha level of 0.05. This statistically
significant result suggests that the intercept is significantly
different from zero. Therefore we reject the null hypothesis
\(H_0: B_{0} = 0\) in favor of the alternative hypothesis
\(H_1: B_{0} \neq 0\)\$.

\hypertarget{season}{%
\subsubsection{Season}\label{season}}

\begin{itemize}
\tightlist
\item
  \textbf{t-value for Season}: 4.589384
\item
  \textbf{p-value for Season}: 0.0006222474
\end{itemize}

The t-value for the year coefficient is significantly positive,
indicating a positive relationship between the year and HR/AB ratio for
Barry Bonds in the dataset analyzed. The p-value associated with this
t-value is much less than 0.05, strongly suggesting that we reject the
null hypothesis \(H_0: B_{1} = 0\) in favor of the alternative
hypothesis \(H_1: B_{1} \neq 0\).

\hypertarget{implications}{%
\subsubsection{Implications}\label{implications}}

The statistical analysis of the year's coefficient reveals that there is
a significant linear relationship between the year and the HR/AB ratio.
This means that the year significantly predicts the HR/AB ratio for
Barry Bonds in the years leading up to 2001, excluding the 2001 season
itself. The positive t-value indicates that as the year increases, so
does the HR/AB ratio, suggesting an improvement in Bonds' performance in
hitting home runs per at-bat attempt over time.

Given the alpha level of 0.05 and the very low p-values obtained for
both the intercept and the year coefficient, our analysis provides
strong statistical evidence to support the conclusion that there was a
significant trend in Barry Bonds' home run rates per at-bat over the
years analyzed.

The R-squared value of 0.6371 in your linear regression model indicates
that approximately 63.71\% of the variance in the dependent variable is
explained by the independent variable.

\textbf{Residual Analysis}

\hypertarget{normality-of-residuals}{%
\subsubsection{Normality of Residuals}\label{normality-of-residuals}}

In our specific analysis concerning Barry Bonds' home run rates (HR/AB)
as a function of the season (year), the normality of residuals implies
that the deviations from the predicted home run rates are random and
follow a normal distribution. This condition supports the premise that
our linear model is appropriately capturing the relationship between the
year and HR/AB without systematic bias.

The evidence supporting the normality of residuals in our analysis comes
from both graphical and statistical methods:

\begin{Shaded}
\begin{Highlighting}[]
\NormalTok{residuals }\OtherTok{\textless{}{-}} \FunctionTok{resid}\NormalTok{(model)}
\NormalTok{residuals\_df }\OtherTok{\textless{}{-}} \FunctionTok{data.frame}\NormalTok{(}\AttributeTok{Residuals =}\NormalTok{ residuals)}

\CommentTok{\# Generate a Q{-}Q plot}
\FunctionTok{ggplot}\NormalTok{(residuals\_df, }\FunctionTok{aes}\NormalTok{(}\AttributeTok{sample =}\NormalTok{ Residuals)) }\SpecialCharTok{+}
  \FunctionTok{stat\_qq}\NormalTok{() }\SpecialCharTok{+}
  \FunctionTok{stat\_qq\_line}\NormalTok{(}\AttributeTok{colour =} \StringTok{"red"}\NormalTok{) }\SpecialCharTok{+}
  \FunctionTok{ggtitle}\NormalTok{(}\StringTok{"Q{-}Q Plot of Residuals"}\NormalTok{) }\SpecialCharTok{+}
  \FunctionTok{theme\_minimal}\NormalTok{()}
\end{Highlighting}
\end{Shaded}

\includegraphics{602_project_files/figure-latex/unnamed-chunk-7-1.pdf}

In our case, the residuals align closely with the reference line in the
Q-Q plot, indicating that their distribution resembles a normal
distribution. This evidence suggests that the residuals from our linear
model do not deviate significantly from normality.

\begin{Shaded}
\begin{Highlighting}[]
\FunctionTok{shapiro.test}\NormalTok{(residuals)}
\end{Highlighting}
\end{Shaded}

\begin{verbatim}
## 
##  Shapiro-Wilk normality test
## 
## data:  residuals
## W = 0.97132, p-value = 0.8938
\end{verbatim}

The Shapiro-Wilk test is a statistical test designed to assess the
normality of a dataset. In our analysis, the Shapiro-Wilk test for the
residuals yields a p-value of 0.8938. Given that this p-value is
significantly greater than the conventional threshold of 0.05, we fail
to reject the null hypothesis that the residuals are normally
distributed. This statistical evidence further substantiates the claim
that the residuals of our model are normal.

\hypertarget{homoscedasticity}{%
\subsubsection{Homoscedasticity}\label{homoscedasticity}}

The examination of homoscedasticity is a critical step in validating the
assumptions underlying a linear regression model. Homoscedasticity
refers to the condition where the residuals (the differences between
observed and predicted values) have constant variance across all levels
of the independent variable(s). This assumption ensures that the model's
predictive accuracy is uniform across the range of the independent
variable, which in this context is the season (year) for Barry Bonds'
home run rates (HR/AB).

\begin{Shaded}
\begin{Highlighting}[]
\NormalTok{fitted\_values }\OtherTok{\textless{}{-}} \FunctionTok{fitted}\NormalTok{(model)}
\NormalTok{homoscedasticity\_df }\OtherTok{\textless{}{-}} \FunctionTok{data.frame}\NormalTok{(}\AttributeTok{Fitted =}\NormalTok{ fitted\_values, }\AttributeTok{Residuals =}\NormalTok{ residuals)}

\FunctionTok{ggplot}\NormalTok{(homoscedasticity\_df, }\FunctionTok{aes}\NormalTok{(}\AttributeTok{x =}\NormalTok{ Fitted, }\AttributeTok{y =}\NormalTok{ Residuals)) }\SpecialCharTok{+}
  \FunctionTok{geom\_point}\NormalTok{() }\SpecialCharTok{+}
  \FunctionTok{geom\_hline}\NormalTok{(}\AttributeTok{yintercept =} \DecValTok{0}\NormalTok{, }\AttributeTok{colour =} \StringTok{"red"}\NormalTok{) }\SpecialCharTok{+}
  \FunctionTok{ggtitle}\NormalTok{(}\StringTok{"Residuals vs. Fitted Values"}\NormalTok{) }\SpecialCharTok{+}
  \FunctionTok{xlab}\NormalTok{(}\StringTok{"Fitted Values"}\NormalTok{) }\SpecialCharTok{+}
  \FunctionTok{ylab}\NormalTok{(}\StringTok{"Residuals"}\NormalTok{) }\SpecialCharTok{+}
  \FunctionTok{theme\_minimal}\NormalTok{()}
\end{Highlighting}
\end{Shaded}

\includegraphics{602_project_files/figure-latex/unnamed-chunk-9-1.pdf}

\begin{Shaded}
\begin{Highlighting}[]
\FunctionTok{bptest}\NormalTok{(model)}
\end{Highlighting}
\end{Shaded}

\begin{verbatim}
## 
##  studentized Breusch-Pagan test
## 
## data:  model
## BP = 0.04849, df = 1, p-value = 0.8257
\end{verbatim}

The Breusch-Pagan (BP) test is a statistical procedure designed to test
for heteroscedasticity --- the presence of non-constant variance in the
error terms of a regression model. In the context of analyzing Barry
Bonds' home run rates (HR/AB) as a function of the season (year), the BP
test's p-value of 0.8257 significantly exceeds the common alpha level
threshold of 0.05. This high p-value indicates that there is
insufficient evidence to reject the null hypothesis of the BP test,
which states that the error variances are homoscedastic.

The even distribution of residuals across the predicted values, as
evidenced by the scatter plot, allows us to conclude that the condition
of homoscedasticity holds for our linear regression model. This finding
is crucial because it means that our model meets another important
assumption of linear regression, reinforcing its validity and the
reliability of its predictions and inferences.

Given that our analysis has confirmed both key assumptions of linear
regression---normality of residuals and homoscedasticity---we can assert
that our linear regression model is valid. This validity implies that
the model is well-founded and that the statistical inferences drawn from
it, such as confidence intervals and hypothesis tests on the regression
coefficients, are based on solid assumptions.

\hypertarget{prediction-of-home-run-rate-in-2001-season}{%
\subsection{Prediction of Home Run Rate in 2001
Season}\label{prediction-of-home-run-rate-in-2001-season}}

\begin{Shaded}
\begin{Highlighting}[]
\NormalTok{new\_data }\OtherTok{\textless{}{-}} \FunctionTok{data.frame}\NormalTok{(}\AttributeTok{season =} \DecValTok{2001}\NormalTok{)}
\NormalTok{predicted\_hrab }\OtherTok{\textless{}{-}} \FunctionTok{predict}\NormalTok{(model, }\AttributeTok{newdata =}\NormalTok{ new\_data, }\AttributeTok{interval=}\StringTok{"predict"}\NormalTok{)}
\FunctionTok{print}\NormalTok{(predicted\_hrab)}
\end{Highlighting}
\end{Shaded}

\begin{verbatim}
##          fit        lwr       upr
## 1 0.09988334 0.06662845 0.1331382
\end{verbatim}

The 95\% prediction interval for Bonds' HRAT in the 2001 season, ranging
from 0.06663 to 0.13314, represents the range within which we would
expect Bonds' HRAT to fall, given the trends observed in the data from
previous seasons. The fact that Bonds' actual HRAT of 0.153400 falls
outside this interval suggests that his performance in 2001 was not only
exceptional but also statistically significant in terms of deviation
from predicted trends.

While the statistical analysis does not directly address whether Barry
Bonds was using steroids in the 2001 season, the significant discrepancy
between the predicted and actual HRATs, and the fact that the actual
HRAT lies outside the 95\% prediction interval, suggest that there were
factors influencing Bonds' performance that year beyond what could be
expected based on historical trends.

In conclusion, while our linear regression model and subsequent analysis
provide valuable insights into Barry Bonds' performance trends leading
up to the 2001 season, they also highlight an extraordinary deviation in
his 2001 performance that warrants further investigation. The
statistical evidence suggests that Bonds' HRAT in 2001 was significantly
higher than expected based on past performance, pointing to the presence
of additional factors that contributed to this anomaly.

It's important to note that statistical analysis alone cannot prove the
use of performance-enhancing substances. Such conclusions would require
corroborative evidence beyond the scope of this statistical model.
However, our analysis does underscore the exceptional nature of Bonds'
2001 season within the context of his career performance trends.

\newpage

\hypertarget{part-ii}{%
\section{Part II}\label{part-ii}}

\hypertarget{introduction}{%
\subsection{Introduction}\label{introduction}}

The ISRUC-Sleep dataset is a resource for researchers interested in the
field of sleep studies. This polysomnographic (PSG) dataset was compiled
to facilitate a wide range of investigations into sleep patterns,
disorders, and the effects of medication on sleep. Comprising data from
100 human adults, including both healthy subjects and those diagnosed
with sleep disorders, the dataset is structured to support various
research objectives and methodologies.

\hypertarget{dataset-overview}{%
\subsubsection{Dataset Overview}\label{dataset-overview}}

The dataset is organized into three main groups, each tailored to
address different research needs:

\begin{enumerate}
\def\labelenumi{\arabic{enumi}.}
\item
  \textbf{General Population Data}: It includes one recording session
  per subject across 100 subjects, offering a broad overview of sleep
  characteristics in a diverse population.The PSG recordings, associated
  with each subject, were visually scored by two human experts. The PSG
  recordings include electrophysiological signals, pneumological
  signals, and another contextual information of the subjects.
\item
  \textbf{Healthy Subjects Data}: This subset focuses on 10 healthy
  individuals, allowing for detailed comparisons between healthy sleep
  patterns and those affected by sleep disorders.
\item
  \textbf{Sleep Stages and Events}: The data contains epoch-by-epoch
  annotations of sleep stages (Awake, NREM stages N1, N2, N3, and REM)
  based on the American Academy of Sleep Medicine (AASM) criteria, as
  well as various sleep-related events and physiological signals (e.g.,
  heart rate, blood-oxygen saturation).
\end{enumerate}

\hypertarget{motivations-for-researching-sleep}{%
\subsubsection{Motivations for Researching
Sleep}\label{motivations-for-researching-sleep}}

Researching sleep is crucial for several reasons, reflecting the central
role of sleep in human health and well-being:

\begin{itemize}
\item
  \textbf{Understanding Sleep Disorders}: With a wide prevalence of
  sleep disorders across the global population, understanding the
  nuances of these conditions is vital for developing effective
  treatments. The ISRUC-Sleep dataset allows researchers to study the
  specific patterns and anomalies associated with different disorders.
\item
  \textbf{Impact on Health}: Sleep has profound effects on physical
  health, mental health, and cognitive function. Research can uncover
  how variations in sleep patterns affect these areas, leading to
  improved guidelines for healthy sleep and interventions for
  sleep-related health issues.
\end{itemize}

In summary, the ISRUC-Sleep dataset is a foundational tool for advancing
our understanding of sleep and its complex interplay with human health.
It enables researchers to explore questions related to sleep physiology,
disorders, treatment effects, and beyond, with the ultimate goal of
enhancing sleep quality and health outcomes for individuals around the
world.

\begin{Shaded}
\begin{Highlighting}[]
\NormalTok{summary\_data }\OtherTok{\textless{}{-}} \FunctionTok{read.csv}\NormalTok{(}\StringTok{"summary\_data.csv"}\NormalTok{)}
\NormalTok{sleep\_stage\_output }\OtherTok{\textless{}{-}} \FunctionTok{read.csv}\NormalTok{(}\StringTok{"sleep\_stage\_output.csv"}\NormalTok{)}

\NormalTok{dataset }\OtherTok{\textless{}{-}} \FunctionTok{c}\NormalTok{(summary\_data, sleep\_stage\_output)}
\NormalTok{dataset }\OtherTok{\textless{}{-}} \FunctionTok{data.frame}\NormalTok{(dataset)}


\NormalTok{dataset }\OtherTok{\textless{}{-}}\NormalTok{ dataset }\SpecialCharTok{\%\textgreater{}\%}
  \FunctionTok{filter}\NormalTok{(Age }\SpecialCharTok{\textgreater{}} \DecValTok{0}\NormalTok{, W }\SpecialCharTok{\textless{}} \DecValTok{400}\NormalTok{) }\SpecialCharTok{\%\textgreater{}\%}
  \FunctionTok{filter}\NormalTok{(Sex }\SpecialCharTok{\%in\%} \FunctionTok{c}\NormalTok{(}\StringTok{"Male"}\NormalTok{, }\StringTok{"Female"}\NormalTok{))}
\end{Highlighting}
\end{Shaded}

\begin{Shaded}
\begin{Highlighting}[]
\FunctionTok{summary}\NormalTok{(dataset)}
\end{Highlighting}
\end{Shaded}

\begin{verbatim}
##        ID             Expert      Date             Height_in      Height_cm    
##  Min.   :  1.00   Min.   :1   Length:73          Min.   :40.0   Min.   :101.0  
##  1st Qu.: 28.00   1st Qu.:1   Class :character   1st Qu.:63.0   1st Qu.:160.0  
##  Median : 52.00   Median :1   Mode  :character   Median :66.0   Median :167.0  
##  Mean   : 51.45   Mean   :1                      Mean   :65.4   Mean   :166.1  
##  3rd Qu.: 77.00   3rd Qu.:1                      3rd Qu.:69.0   3rd Qu.:174.0  
##  Max.   :100.00   Max.   :1                      Max.   :77.0   Max.   :195.0  
##                                                                                
##    Weight_lbs      Weight_kg           Age            Sex           
##  Min.   :  0.0   Min.   : 53.00   Min.   :20.00   Length:73         
##  1st Qu.:161.0   1st Qu.: 73.00   1st Qu.:37.00   Class :character  
##  Median :176.0   Median : 80.00   Median :50.00   Mode  :character  
##  Mean   :175.8   Mean   : 80.90   Mean   :49.23                     
##  3rd Qu.:196.0   3rd Qu.: 89.25   3rd Qu.:61.00                     
##  Max.   :254.0   Max.   :115.00   Max.   :85.00                     
##                  NA's   :1                                          
##   Total_Epoch        Min_SpO2         Min_HR         Max_SpO2     
##  Min.   : 341.0   Min.   : 8.00   Min.   :24.00   Min.   : 94.00  
##  1st Qu.: 809.0   1st Qu.:59.00   1st Qu.:47.00   1st Qu.: 97.00  
##  Median : 872.0   Median :84.00   Median :57.00   Median : 98.00  
##  Mean   : 848.2   Mean   :69.23   Mean   :54.21   Mean   : 97.92  
##  3rd Qu.: 897.0   3rd Qu.:88.00   3rd Qu.:62.00   3rd Qu.: 99.00  
##  Max.   :1032.0   Max.   :97.00   Max.   :74.00   Max.   :100.00  
##                                                                   
##      Max_HR       Average_SpO2     Average_HR    BPOS_Summary      
##  Min.   : 76.0   Min.   :86.76   Min.   :47.38   Length:73         
##  1st Qu.: 97.0   1st Qu.:93.28   1st Qu.:61.13   Class :character  
##  Median :107.0   Median :94.84   Median :67.86   Mode  :character  
##  Mean   :126.9   Mean   :94.43   Mean   :68.82                     
##  3rd Qu.:125.0   3rd Qu.:95.86   3rd Qu.:76.29                     
##  Max.   :255.0   Max.   :99.30   Max.   :94.03                     
##                                                                    
##  Stage_Summary      Events_Summary       Row.Index           W        
##  Length:73          Length:73          Min.   :  0.0   Min.   :  5.0  
##  Class :character   Class :character   1st Qu.: 54.0   1st Qu.: 95.0  
##  Mode  :character   Mode  :character   Median :102.0   Median :149.0  
##                                        Mean   :100.9   Mean   :166.9  
##                                        3rd Qu.:152.0   3rd Qu.:233.0  
##                                        Max.   :198.0   Max.   :382.0  
##                                                                       
##        N3              N2              R             N1            n2     
##  Min.   : 42.0   Min.   :139.0   Min.   :  8   Min.   : 11   Min.   : NA  
##  1st Qu.:134.0   1st Qu.:208.0   1st Qu.: 84   1st Qu.: 74   1st Qu.: NA  
##  Median :171.0   Median :278.0   Median :121   Median :102   Median : NA  
##  Mean   :177.4   Mean   :276.4   Mean   :120   Mean   :107   Mean   :NaN  
##  3rd Qu.:205.0   3rd Qu.:338.0   3rd Qu.:139   3rd Qu.:137   3rd Qu.: NA  
##  Max.   :417.0   Max.   :432.0   Max.   :330   Max.   :245   Max.   : NA  
##                                                              NA's   :73   
##        U               N     
##  Min.   :1.000   Min.   :2   
##  1st Qu.:1.000   1st Qu.:2   
##  Median :5.000   Median :2   
##  Mean   :4.625   Mean   :2   
##  3rd Qu.:7.250   3rd Qu.:2   
##  Max.   :9.000   Max.   :2   
##  NA's   :65      NA's   :72
\end{verbatim}

The summary stats show that there are still some incomplete rows in our
dataset (for example, there are still some people without a recorded
weight), however the data that we are focusing on for the analysis
appears to be complete now. These values are Age which goes from 20 to
85 with a median of 50 and a mean of 49.23 as well as Wake Time which is
recorded as number of 30 second epochs. The values for Wake Time go from
5 to 382 (2.5 minutes to 191 minutes) with a median of 149 (74.5
minutes) and a mean of 166.9 (83.45 minutes).

For a more comprehensive exploration of the ISRUC-Sleep dataset, let's
delve into additional visualizations that can illuminate various aspects
of the data. These visualizations will help uncover patterns, trends,
and correlations within the dataset, providing deeper insights into
sleep behavior, physiological measures, and their interactions with
demographic factors like age.

\begin{Shaded}
\begin{Highlighting}[]
\NormalTok{physiological\_data }\OtherTok{\textless{}{-}}\NormalTok{ dataset[, }\FunctionTok{c}\NormalTok{(}\StringTok{"Average\_SpO2"}\NormalTok{, }\StringTok{"Average\_HR"}\NormalTok{, }\StringTok{"Min\_SpO2"}\NormalTok{,}
                                  \StringTok{"Min\_HR"}\NormalTok{, }\StringTok{"Max\_SpO2"}\NormalTok{, }\StringTok{"Max\_HR"}\NormalTok{, }\StringTok{"W"}\NormalTok{, }\StringTok{"N3"}\NormalTok{,}
                                  \StringTok{"N2"}\NormalTok{,}\StringTok{"R"}\NormalTok{,}\StringTok{"N1"}\NormalTok{)]}
\NormalTok{cor\_matrix }\OtherTok{\textless{}{-}} \FunctionTok{cor}\NormalTok{(physiological\_data, }\AttributeTok{use =} \StringTok{"complete.obs"}\NormalTok{)}

\FunctionTok{corrplot}\NormalTok{(cor\_matrix, }\AttributeTok{method =} \StringTok{"color"}\NormalTok{, }\AttributeTok{type =} \StringTok{"upper"}\NormalTok{, }\AttributeTok{order =} \StringTok{"hclust"}\NormalTok{,}
         \AttributeTok{tl.col =} \StringTok{"black"}\NormalTok{, }\AttributeTok{mar=}\FunctionTok{c}\NormalTok{(}\DecValTok{0}\NormalTok{,}\DecValTok{0}\NormalTok{,}\DecValTok{2}\NormalTok{,}\DecValTok{0}\NormalTok{),}
         \AttributeTok{title =} \StringTok{"Correlation of Physiological Measures"}\NormalTok{)}
\end{Highlighting}
\end{Shaded}

\includegraphics{602_project_files/figure-latex/unnamed-chunk-14-1.pdf}

A correlation matrix visualization can help identify relationships
between continuous variables such as age, wake time, average heart rate
(HR), and average blood-oxygen saturation (SpO2). We can see that there
are song strong correlations based on the minimums of various
physiological factors.

\begin{Shaded}
\begin{Highlighting}[]
\FunctionTok{ggplot}\NormalTok{(dataset, }\FunctionTok{aes}\NormalTok{(}\AttributeTok{x =}\NormalTok{ Age)) }\SpecialCharTok{+} 
  \FunctionTok{geom\_histogram}\NormalTok{(}\AttributeTok{binwidth =} \DecValTok{5}\NormalTok{, }\AttributeTok{fill =} \StringTok{"cornflowerblue"}\NormalTok{, }\AttributeTok{color =} \StringTok{"black"}\NormalTok{) }\SpecialCharTok{+} 
  \FunctionTok{facet\_wrap}\NormalTok{(}\SpecialCharTok{\textasciitilde{}}\NormalTok{Sex) }\SpecialCharTok{+} 
  \FunctionTok{labs}\NormalTok{(}\AttributeTok{title =} \StringTok{"Age Distribution by Sex"}\NormalTok{, }\AttributeTok{x =} \StringTok{"Age"}\NormalTok{, }\AttributeTok{y =} \StringTok{"Count"}\NormalTok{) }\SpecialCharTok{+}
  \FunctionTok{theme\_minimal}\NormalTok{()}
\end{Highlighting}
\end{Shaded}

\includegraphics{602_project_files/figure-latex/unnamed-chunk-15-1.pdf}

\begin{Shaded}
\begin{Highlighting}[]
\FunctionTok{ggplot}\NormalTok{(dataset, }\FunctionTok{aes}\NormalTok{(}\AttributeTok{x =}\NormalTok{ W)) }\SpecialCharTok{+} 
  \FunctionTok{geom\_histogram}\NormalTok{(}\AttributeTok{binwidth =} \DecValTok{10}\NormalTok{, }\AttributeTok{fill =} \StringTok{"\#ff9100"}\NormalTok{, }\AttributeTok{color =} \StringTok{"black"}\NormalTok{) }\SpecialCharTok{+} \FunctionTok{facet\_wrap}\NormalTok{(}\SpecialCharTok{\textasciitilde{}}\NormalTok{Sex) }\SpecialCharTok{+} 
  \FunctionTok{labs}\NormalTok{(}\AttributeTok{title =} \StringTok{"Distribution of Wake Time"}\NormalTok{, }\AttributeTok{x =} \StringTok{"Wake Time (epochs)"}\NormalTok{,}
       \AttributeTok{y =} \StringTok{"Frequency"}\NormalTok{) }\SpecialCharTok{+} 
  \FunctionTok{theme\_minimal}\NormalTok{()}
\end{Highlighting}
\end{Shaded}

\includegraphics{602_project_files/figure-latex/unnamed-chunk-15-2.pdf}

These plots show the distribution of ages and wake times within the
dataset, separated by sex, concluding that there are more older males
than there are women.

\begin{Shaded}
\begin{Highlighting}[]
\FunctionTok{ggplot}\NormalTok{(dataset, }\FunctionTok{aes}\NormalTok{(}\AttributeTok{x =}\NormalTok{ Sex, }\AttributeTok{y =}\NormalTok{ W, }\AttributeTok{fill =}\NormalTok{ Sex)) }\SpecialCharTok{+} 
  \FunctionTok{geom\_boxplot}\NormalTok{() }\SpecialCharTok{+} 
  \FunctionTok{labs}\NormalTok{(}\AttributeTok{title =} \StringTok{"Wake Time by Sex"}\NormalTok{, }\AttributeTok{x =} \StringTok{"Sex"}\NormalTok{, }\AttributeTok{y =} \StringTok{"Wake Time (epochs)"}\NormalTok{) }\SpecialCharTok{+} 
  \FunctionTok{scale\_fill\_brewer}\NormalTok{(}\AttributeTok{palette =} \StringTok{"Set2"}\NormalTok{) }\SpecialCharTok{+} 
  \FunctionTok{theme\_minimal}\NormalTok{()}
\end{Highlighting}
\end{Shaded}

\includegraphics{602_project_files/figure-latex/unnamed-chunk-16-1.pdf}

A boxplot can illustrate differences in wake time across sexes,
highlighting any potential gender differences in sleep patterns, such
that on average males have a higher wake time than women.

\begin{Shaded}
\begin{Highlighting}[]
\FunctionTok{ggplot}\NormalTok{(dataset, }\FunctionTok{aes}\NormalTok{(}\AttributeTok{x =}\NormalTok{ Average\_HR, }\AttributeTok{y =}\NormalTok{ Average\_SpO2, }\AttributeTok{color =}\NormalTok{ Age)) }\SpecialCharTok{+} 
  \FunctionTok{geom\_point}\NormalTok{(}\AttributeTok{alpha =} \FloatTok{0.7}\NormalTok{) }\SpecialCharTok{+} 
  \FunctionTok{scale\_color\_gradient}\NormalTok{(}\AttributeTok{low =} \StringTok{"blue"}\NormalTok{, }\AttributeTok{high =} \StringTok{"red"}\NormalTok{) }\SpecialCharTok{+} 
  \FunctionTok{labs}\NormalTok{(}\AttributeTok{title =} \StringTok{"Average HR vs. Average SpO2, Colored by Age"}\NormalTok{,}
       \AttributeTok{x =} \StringTok{"Average Heart Rate (bpm)"}\NormalTok{, }\AttributeTok{y =} \StringTok{"Average SpO2 (\%)"}\NormalTok{) }\SpecialCharTok{+} 
  \FunctionTok{theme\_minimal}\NormalTok{()}
\end{Highlighting}
\end{Shaded}

\includegraphics{602_project_files/figure-latex/unnamed-chunk-17-1.pdf}

This scatter plot can help visualize the relationship between average
heart rate and average SpO2, with points colored by age to see if
there's any age-related pattern, which as we can see the color gets
slightly brighter the lower we get on both average hr and average spO2.

\hypertarget{model-building-1}{%
\subsection{Model Building}\label{model-building-1}}

The statistical model we intend to develop,
\(\hat{W_{i}} = \alpha + \beta \times Age_{i} + \epsilon_{i}\), where
\(\hat{W_{i}}\) symbolizes the predicted wake times for the \(i^{th}\)
individual, and \(Age_{i}\) represents the age of the individual, aims
to shed light on the relationship between age and wake times. By
quantifying the trend across different ages, this linear regression
model acts as a foundational tool for investigating potential
significant changes in wake times across the lifespan. Such changes, if
detected, could indicate underlying physiological, psychological, or
lifestyle shifts associated with aging. This model's ability to
highlight deviations from expected patterns provides a quantitative
basis for further exploration into the factors influencing sleep
patterns and the potential impact of age on sleep quality and duration.

\begin{Shaded}
\begin{Highlighting}[]
\NormalTok{model }\OtherTok{\textless{}{-}} \FunctionTok{lm}\NormalTok{(W }\SpecialCharTok{\textasciitilde{}}\NormalTok{ Age, }\AttributeTok{data =}\NormalTok{ dataset)}
\NormalTok{model}\SpecialCharTok{$}\NormalTok{coef}
\end{Highlighting}
\end{Shaded}

\begin{verbatim}
## (Intercept)         Age 
##    11.36195     3.16015
\end{verbatim}

The linear regression equation representing the model is
\(\hat{W_{i}} = 11.36195 + 3.16015*Age_{i}\)

\begin{Shaded}
\begin{Highlighting}[]
\FunctionTok{ggplot}\NormalTok{(dataset, }\FunctionTok{aes}\NormalTok{(}\AttributeTok{x =}\NormalTok{ Age, }\AttributeTok{y =}\NormalTok{ W)) }\SpecialCharTok{+} 
  \FunctionTok{geom\_point}\NormalTok{() }\SpecialCharTok{+} 
  \FunctionTok{geom\_smooth}\NormalTok{(}\AttributeTok{method =} \StringTok{"lm"}\NormalTok{, }\AttributeTok{se=}\ConstantTok{FALSE}\NormalTok{,}\AttributeTok{color =} \StringTok{"\#ff7700"}\NormalTok{) }\SpecialCharTok{+} 
  \FunctionTok{labs}\NormalTok{(}\AttributeTok{title =} \StringTok{"Wake Time vs. Age"}\NormalTok{, }\AttributeTok{x =} \StringTok{"Age"}\NormalTok{, }\AttributeTok{y =} \StringTok{"Wake Time (epochs)"}\NormalTok{)}
\end{Highlighting}
\end{Shaded}

\includegraphics{602_project_files/figure-latex/unnamed-chunk-19-1.pdf}

This scatter plot, complemented by a linear regression line, explores
the relationship between wake time and age. Points represent individual
subjects, plotting their wake time against their age. The linear
regression line (smoothed line) provides a visual estimate of the trend,
showing whether wake time tends to increases as age progresses.

Importantly, to ensure the integrity of the conclusions drawn from this
model, we will verify the assumptions underlying linear regression
analysis, including linearity, independence of errors, homoscedasticity
(constant variance of errors), and normality of error terms. These
checks are fundamental to confirming that our model is accurately
specified and that the inferences and predictions derived from it are
reliable.

\hypertarget{model-assumption-and-validation-1}{%
\subsection{Model Assumption and
Validation}\label{model-assumption-and-validation-1}}

\textbf{\emph{Correlation Coefficient Check}}:

\begin{Shaded}
\begin{Highlighting}[]
\FunctionTok{cor}\NormalTok{(dataset}\SpecialCharTok{$}\NormalTok{W, dataset}\SpecialCharTok{$}\NormalTok{Age)}
\end{Highlighting}
\end{Shaded}

\begin{verbatim}
## [1] 0.5353568
\end{verbatim}

A correlation coefficient of 0.5353568 indicates a positive linear
relationship between the age and amount of time spent awake during
sleep.

\textbf{\emph{Significance of Coefficient Estimates}}

In the context of examining the importance of the coefficient estimate
for age in predicting wake times, we can define the null hypothesis
(\(H_0\)) and the alternative hypothesis (\(H_1\)) as follows, with a
significance level (\(\alpha\)) of 0.05:

Null hypothesis: \(H_{0}: \beta_{1} = 0\), \(H_{0}: \beta_{0} = 0\)

Alternative hypothesis: \(H_{A}: \beta_{1} \neq 0\),
\(H_{A}: \beta_{0} \neq 0\)

\begin{Shaded}
\begin{Highlighting}[]
\FunctionTok{summary}\NormalTok{(model)}
\end{Highlighting}
\end{Shaded}

\begin{verbatim}
## 
## Call:
## lm(formula = W ~ Age, data = dataset)
## 
## Residuals:
##     Min      1Q  Median      3Q     Max 
## -159.29  -61.09  -20.61   48.59  194.31 
## 
## Coefficients:
##             Estimate Std. Error t value Pr(>|t|)    
## (Intercept)  11.3620    30.5293   0.372    0.711    
## Age           3.1601     0.5917   5.341 1.06e-06 ***
## ---
## Signif. codes:  0 '***' 0.001 '**' 0.01 '*' 0.05 '.' 0.1 ' ' 1
## 
## Residual standard error: 78.04 on 71 degrees of freedom
## Multiple R-squared:  0.2866, Adjusted R-squared:  0.2766 
## F-statistic: 28.52 on 1 and 71 DF,  p-value: 1.061e-06
\end{verbatim}

\begin{Shaded}
\begin{Highlighting}[]
\FunctionTok{coef}\NormalTok{(}\FunctionTok{summary}\NormalTok{(model))[, }\StringTok{"t value"}\NormalTok{]}
\end{Highlighting}
\end{Shaded}

\begin{verbatim}
## (Intercept)         Age 
##   0.3721653   5.3408207
\end{verbatim}

\begin{Shaded}
\begin{Highlighting}[]
\FunctionTok{coef}\NormalTok{(}\FunctionTok{summary}\NormalTok{(model))[, }\StringTok{"Pr(\textgreater{}|t|)"}\NormalTok{]}
\end{Highlighting}
\end{Shaded}

\begin{verbatim}
##  (Intercept)          Age 
## 7.108783e-01 1.060644e-06
\end{verbatim}

\hypertarget{intercept-1}{%
\subsubsection{Intercept}\label{intercept-1}}

\begin{itemize}
\tightlist
\item
  \textbf{t-value for Intercept}: 0.3721653
\item
  \textbf{p-value for Intercept}: 0.7108783
\end{itemize}

The intercept's t-value is not significantly far from zero, and the
corresponding p-value is much greater than the alpha level of 0.05. This
result suggests that the intercept is not statistically significantly
different from zero at the conventional significance levels. Therefore,
we fail to reject the null hypothesis \(H_0: \beta_{0} = 0\) in favor of
the alternative hypothesis \(H_1: \beta_{0} \neq 0\).

\hypertarget{age}{%
\subsubsection{Age}\label{age}}

\begin{itemize}
\tightlist
\item
  \textbf{t-value for Age}: 5.3408207
\item
  \textbf{p-value for Age}: 0.000001060644
\end{itemize}

The t-value for the age coefficient is significantly positive,
indicating a positive relationship between age and wake times in the
dataset analyzed. The p-value associated with this t-value is much less
than 0.05, strongly suggesting that we reject the null hypothesis
\(H_0: \beta_{1} = 0\) in favor of the alternative hypothesis
\(H_1: \beta_{1} \neq 0\).

\hypertarget{implications-1}{%
\subsubsection{Implications}\label{implications-1}}

The statistical analysis of the age coefficient reveals that there is a
significant linear relationship between age and wake times. This means
that age significantly predicts wake times, with the positive t-value
indicating that as age increases, so does the wake time, suggesting an
association between older age and longer wake times.

Given the alpha level of 0.05 and the very low p-value obtained for the
age coefficient, our analysis provides strong statistical evidence to
support the conclusion that age is a significant predictor of wake
times. The intercept not being significantly different from zero
suggests that the baseline wake time (at age zero, hypothetically
speaking) is not statistically distinguishable from zero in this model's
context.

The R-squared value of 0.2766 in our linear regression model, derived
from real-world data, indicates that approximately 27.66\% of the
variance in the dependent variable (wake times) is explained by the
independent variable (age). This figure, while lower than ideal
theoretical models, is not uncommon in studies involving complex human
behaviors and physiological responses, where multiple factors beyond age
might influence the outcome. The value underscores the inherent
complexity and variability in real-world datasets, highlighting that
while age is a significant predictor of wake times, a substantial
portion of the variance remains unexplained by this model alone.

\textbf{Residual Analysis}

\hypertarget{normality-of-residuals-1}{%
\subsubsection{Normality of Residuals}\label{normality-of-residuals-1}}

In our specific analysis concerning wake time as a function of age, the
normality of residuals implies that the deviations from the predicted
wake times are random and follow a normal distribution. This condition
supports the premise that our linear model is appropriately capturing
the relationship between age and wake time without systematic bias.

The evidence supporting the normality of residuals in our analysis comes
from both graphical and statistical methods:

\begin{Shaded}
\begin{Highlighting}[]
\NormalTok{residuals }\OtherTok{\textless{}{-}} \FunctionTok{resid}\NormalTok{(model)}
\NormalTok{residuals\_df }\OtherTok{\textless{}{-}} \FunctionTok{data.frame}\NormalTok{(}\AttributeTok{Residuals =}\NormalTok{ residuals)}

\CommentTok{\# Generate a Q{-}Q plot}
\FunctionTok{ggplot}\NormalTok{(residuals\_df, }\FunctionTok{aes}\NormalTok{(}\AttributeTok{sample =}\NormalTok{ Residuals)) }\SpecialCharTok{+}
  \FunctionTok{stat\_qq}\NormalTok{() }\SpecialCharTok{+}
  \FunctionTok{stat\_qq\_line}\NormalTok{(}\AttributeTok{colour =} \StringTok{"red"}\NormalTok{) }\SpecialCharTok{+}
  \FunctionTok{ggtitle}\NormalTok{(}\StringTok{"Q{-}Q Plot of Residuals"}\NormalTok{) }\SpecialCharTok{+}
  \FunctionTok{theme\_minimal}\NormalTok{()}
\end{Highlighting}
\end{Shaded}

\includegraphics{602_project_files/figure-latex/unnamed-chunk-22-1.pdf}

In our case, the residuals deviate slightly with the reference line in
the Q-Q plot, indicating that their distribution might not be a normal
distribution. This evidence suggests that the residuals from our linear
model do not deviate significantly from normality but to be certain we
can perform a Shapiro Wilks test with an alpha of 0.01.

\begin{Shaded}
\begin{Highlighting}[]
\FunctionTok{shapiro.test}\NormalTok{(residuals)}
\end{Highlighting}
\end{Shaded}

\begin{verbatim}
## 
##  Shapiro-Wilk normality test
## 
## data:  residuals
## W = 0.96323, p-value = 0.03179
\end{verbatim}

With a p-value of 0.03179 from the Shapiro-Wilk test and an alpha level
of 0.01, we do not reject the null hypothesis that the data is normally
distributed. The p-value is greater than the alpha level, indicating
that there is not enough evidence to conclude the data deviates from
normality at the 1\% significance level. This result supports the
assumption of normality for the residuals in our analysis, suggesting
that our statistical model's underlying assumptions are met, enhancing
the reliability of the conclusions drawn from it in the context of
real-world data variability.

\hypertarget{homoscedasticity-1}{%
\subsubsection{Homoscedasticity}\label{homoscedasticity-1}}

The examination of homoscedasticity is a critical step in validating the
assumptions underlying a linear regression model. Homoscedasticity
refers to the condition where the residuals (the differences between
observed and predicted values) have constant variance across all levels
of the independent variable(s). This assumption ensures that the model's
predictive accuracy is uniform across the range of the independent
variable.

\begin{Shaded}
\begin{Highlighting}[]
\NormalTok{fitted\_values }\OtherTok{\textless{}{-}} \FunctionTok{fitted}\NormalTok{(model)}
\NormalTok{homoscedasticity\_df }\OtherTok{\textless{}{-}} \FunctionTok{data.frame}\NormalTok{(}\AttributeTok{Fitted =}\NormalTok{ fitted\_values, }\AttributeTok{Residuals =}\NormalTok{ residuals)}

\FunctionTok{ggplot}\NormalTok{(homoscedasticity\_df, }\FunctionTok{aes}\NormalTok{(}\AttributeTok{x =}\NormalTok{ Fitted, }\AttributeTok{y =}\NormalTok{ Residuals)) }\SpecialCharTok{+}
  \FunctionTok{geom\_point}\NormalTok{() }\SpecialCharTok{+}
  \FunctionTok{geom\_hline}\NormalTok{(}\AttributeTok{yintercept =} \DecValTok{0}\NormalTok{, }\AttributeTok{colour =} \StringTok{"red"}\NormalTok{) }\SpecialCharTok{+}
  \FunctionTok{ggtitle}\NormalTok{(}\StringTok{"Residuals vs. Fitted Values"}\NormalTok{) }\SpecialCharTok{+}
  \FunctionTok{xlab}\NormalTok{(}\StringTok{"Fitted Values"}\NormalTok{) }\SpecialCharTok{+}
  \FunctionTok{ylab}\NormalTok{(}\StringTok{"Residuals"}\NormalTok{) }\SpecialCharTok{+}
  \FunctionTok{theme\_minimal}\NormalTok{()}
\end{Highlighting}
\end{Shaded}

\includegraphics{602_project_files/figure-latex/unnamed-chunk-24-1.pdf}

\begin{Shaded}
\begin{Highlighting}[]
\FunctionTok{bptest}\NormalTok{(model)}
\end{Highlighting}
\end{Shaded}

\begin{verbatim}
## 
##  studentized Breusch-Pagan test
## 
## data:  model
## BP = 0.83438, df = 1, p-value = 0.361
\end{verbatim}

The Breusch-Pagan (BP) test is a statistical procedure designed to test
for heteroscedasticity --- the presence of non-constant variance in the
error terms of a regression model. The BP test's p-value of 0.361
significantly exceeds the common alpha level threshold of 0.05. This
high p-value indicates that there is insufficient evidence to reject the
null hypothesis of the BP test, which states that the error variances
are homoscedastic.

The even distribution of residuals across the predicted values, as
evidenced by the scatter plot, allows us to conclude that the condition
of homoscedasticity holds for our linear regression model. This finding
is crucial because it means that our model meets another important
assumption of linear regression, reinforcing its validity and the
reliability of its predictions and inferences.

Given that our analysis has confirmed both key assumptions of linear
regression---normality of residuals and homoscedasticity---we can assert
that our linear regression model is valid. This validity implies that
the model is well-founded and that the statistical inferences drawn from
it, such as confidence intervals and hypothesis tests on the regression
coefficients, are based on solid assumptions.

\hypertarget{references}{%
\subsection{References}\label{references}}

Khalighi Sirvan, Teresa Sousa, José Moutinho Santos, and Urbano Nunes.
``ISRUC-Sleep: A comprehensive public dataset for sleep
researchers.``Computer methods and programs in biomedicine 124 (2016):
180-192.

\end{document}
